\documentclass[11pt]{article}
\usepackage[margin=0.75in]{geometry}
\usepackage{graphicx}
\usepackage{siunitx}
\usepackage{amsmath}
\usepackage{float}
\usepackage{caption}
\usepackage{subcaption}
\usepackage{circuitikz}


\title{Lab 2: Transistor Characteristics}
\author{Jason Jain \\ 
        Oregon State University, Department of Physics \\
        Electronics Laboratory}
\date{\today}

\begin{document}

\maketitle

\section{Transistor Characteristics}

\begin{figure}[H]
    \centering
    \begin{circuitikz}[scale=1]
        % Transistor
        \draw (0,0) node[npn] (T) {};
        
        % Collector circuit
        \draw (T.collector) to[leDo, invert] (0,2) to[short] (0,2.5) node[above] {+5V};
        
        % Base circuit
        \draw (-2,2.5) node[above] {+5V} to[R, l=10k$\Omega$] (-2,0) to[short] (T.base);
        \draw (-2,0) to[short] (-2,-1.5) node[ground] {};
        

        % Emitter
        \draw (T.emitter) to[R, l=220$\Omega$] (0,-1.5) node[ground] {};
    \end{circuitikz}
    \caption{Transistor measurement circuit for determining current gain and base-emitter characteristics.}
    \label{fig:transistor_circuit}
\end{figure}

\begin{figure}[H]
    \centering
    \includegraphics[width=0.6\textwidth]{transistor_current_characteristics.png}
    \caption{Linear relationship between base current and collector current demonstrates transistor amplification.}
    \label{fig:transistor_current}
\end{figure}

\begin{figure}[H]
    \centering
    \includegraphics[width=0.6\textwidth]{collector_current_vs._base-emitter_voltage.png}
    \caption{Voltage vs current shows beginning of an exponential diode curve relationship, but outliers and issues in measurement seem to affect the shape of the curve. }
    \label{fig:base_emitter}
\end{figure}



Using the circuit in Figure 1, we measure a linear relationship between collector current $I_C$ and base current $I_B$. The slope of the line in Figure 2 gives us the current gain factor $\beta=128.66$, obtained from the best fit line in the graph. The high collector current allows us to use the transistor as an amplifier, amd demonstrates the current amplification property.

We were unsure why both curves have a negative slope, but the shape of the data looks correct, overall. The diode curve looks similar to previous exponential diode curves, but behaves odd as the voltage approaches the extreme ends on either side. A mistake in measurement could have flipped the data values. We brought it up with Professor Graham, but he was also unsure. 




\newpage


\section{Logic Gates}



We built an SR Latch and an AND gate to create a simple 1-bit computer. The AND gate outputs a $1$ when both the buttons are pressed. That result is sent to one of the inputs of the SR latch, causing its metastable state to switch, and storing the result of our AND operation in 1-bit memory. The latch persists its state even when the switch isn't pressed, but clears its state when power is cut. 

\begin{figure}[H]
    \centering
    \includegraphics[width=0.5\textwidth]{complete_circuit.jpeg}
    \caption{Complete circuit showing AND gate and SR latch components before connection.}
    \label{fig:complete_circuit}
\end{figure} 

\begin{figure}[H]
    \centering
    \includegraphics[width=0.5\textwidth]{and_gate_to_latch.jpeg}
    \caption{Demonstration of write to memory operation - AND gate output connected to SR latch input.}
    \label{fig:memory_write}
\end{figure}

\newpage

\subsection{AND Gate}

The AND gate uses two cascaded NPN transistors in series. Both switches A and B must be closed to provide base current to both transistors, allowing current to flow through the LED. AND gates are elementary logic gates that function as building blocks of computers, along with OR, NOT, etc. 


\begin{figure}[H]
    \centering
    \includegraphics[width=0.5\textwidth]{and_gate.jpeg}
    \caption{AND gate circuit implementation. Output LED is lit when both buttons pressed.}
    \label{fig:and_gate}
\end{figure}



\subsubsection{AND Gate Truth Table}

\begin{center}
\begin{tabular}{|c|c||c|}
\hline
\textbf{A} & \textbf{B} & \textbf{Q} \\
\hline
0 & 0 & 0 \\
0 & 1 & 0 \\
1 & 0 & 0 \\
1 & 1 & 1 \\
\hline
\end{tabular}
\end{center}

\newpage


\subsection{SR-Latch}


The SR latch uses two cross-coupled NPN transistors where each transistor's base connects to the opposite transistor's collector through 10k$\Omega$ resistors. SR latches can function as building blocks of computer RAM. The cross-coupled design creates a metastable circuit with two stable states, storing one bit of information indefinitely (while powered). 


\begin{figure}[H]
    \centering
    \includegraphics[width=0.5\textwidth]{sr_toggled.jpeg}
    \caption{SR-Latch circuit toggled to $1$ state.}
    \label{fig:sr_latch}
\end{figure}

\subsubsection{SR-Latch Truth Table}

\begin{center}
\begin{tabular}{|c|c||c|c|}
\hline
\textbf{S} & \textbf{R} & \textbf{Q} & \textbf{Q'} \\
\hline
0 & 0 & Hold & Hold \\
1 & 0 & 1 & 0 \\
0 & 1 & 0 & 1 \\
1 & 1 & Invalid & Invalid \\
\hline
\end{tabular}
\end{center}


\end{document}