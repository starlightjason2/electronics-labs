\documentclass[11pt]{article}
\usepackage[margin=0.75in]{geometry}
\usepackage{graphicx}
\usepackage{siunitx}
\usepackage{amsmath}
\usepackage{float}
\usepackage{caption}

\title{Lab 3: Capacitors and Time-Dependent Signals}
\author{Jason Jain \\
        Oregon State University, Department of Physics \\
        Electronics Laboratory}
\date{\today}

\graphicspath{{output/}}

\begin{document}

\maketitle

\section{Time-dependent analysis of RC circuits}

The resistance is $\qty{997}{\ohm}$ and the capacitance is $\qty{73.5}{\farad}$

% \ref{fig:low_pass_1kHz}
\begin{figure}[H]
    \centering
    \includegraphics[width=0.85\textwidth]{low_pass_square_1kHz.png}
    \caption{Measurement of time constact $\tau = RC$ by determining the time for the output to drop to $1/e$ of the
    maximum (in green)and to rise to $1 - (1/e)$ of the maximum (in red). Both values are equal to $\tau = \qty{62}{\micro\second}$.}
    \label{fig:low_pass_1kHz}
\end{figure}

\newpage

\section{The RC Integrator (Low-Pass Filter)}

- Use the circuit of Part 1 and apply a 20 kHz square-wave signal.

- Explain how the observed waveform is consistent with the concept of an RC circuit behaving as an integrator.

- Over what frequency range does the circuit behave as an integrator, that is, is capable of producing a triangle-wave output from a square-wave input? Explain why this circuit is also known as a low-pass filter.


\begin{figure}[H]
    \centering
    \includegraphics[width=0.85\textwidth]{low_pass_square_20kHz.png}
    \caption{RC integrator (low-pass filter): voltage vs time (20 kHz square input).}
    \label{fig:low_pass_20kHz}
\end{figure}

\begin{figure}[H]
    \centering
    \includegraphics[width=0.85\textwidth]{low_pass_triangle_200kHz.png}
    \caption{RC integrator (low-pass filter): voltage vs time (200 kHz triangle input).}
    \label{fig:low_pass_20kHz}
\end{figure}


\newpage

\section{The CR Differentiator (High-Pass Filter)}

- Explain how the observed waveform is consistent with the concept of an CR circuit behaving as an differentiator.


- Vary the frequency of the square wave input from 1 Hz to 1 MHz and describe the behavior of this circuit. Does it ever appear to behave as an integrator or a differentiator? Explain why this circuit is also known as a high-pass filter.



\begin{figure}[H]
    \centering
    \includegraphics[width=0.85\textwidth]{high_pass_square_510Hz.png}
    \caption{High-pass filter: voltage vs time (510 Hz square input).}
    \label{fig:high_pass_510Hz}
\end{figure}



\begin{figure}[H]
    \centering
    \includegraphics[width=0.85\textwidth]{high_pass_triangle_200Hz.png}
    \caption{RC integrator (high-pass filter): voltage vs time (200 Hz triangle input).}
    \label{fig:low_pass_20kHz}
\end{figure}

\newpage

\section{Frequency response of both low-pass and high-pass filters}

The goal of this experiment is to measure the frequency dependence of each filter's response to a sine wave input. Measure both the amplitude and phase shift of the filter output relative to the input signal. Set the function generator to provide a sine wave with no offset. Manually vary the frequency from \qty{1}{\hertz} to \qty{1}{\mega\hertz} and observe the variation in amplitude and phase of the output relative to the input. Make a series of at least 20 measurements over the full frequency range. Because you will be plotting your data on a logarithmic plot, make at least two measurements per decade of frequency.

Determine the transmission function $A(\nu)$ by dividing the output amplitude by the input amplitude. Be sure to measure the input amplitude from the function generator at each frequency, because the combination of your circuit and the limitations of the generator may lead to a signal that changes in amplitude with frequency. Measure the phase difference $\phi(\nu)$ between the output and input signals. A good point on the waveform to use for such measurements is the point at which the trace crosses 0~V (i.e.\ ground). If the period of the input signal is $T$ and the displacement of the output signal zero-crossing from the input signal zero-crossing is $t$, then the phase difference is $\phi = 2\pi t / T$. If the output signal zero-crossing occurs after the input signal zero-crossing, then that is a phase lag or a negative phase.

For each filter, plot the data and theoretical curves together. For the amplitude plots, use the decibel (dB) scale $20\log_{10} A(\nu)$ for the vertical axis and $\log \nu$ for the horizontal axis (as in Fig.~4). 

\begin{figure}[H]
    \centering
    \includegraphics[width=0.85\textwidth]{low_pass_bode_plot.png}
    \caption{Low-pass filter: bode plot}
    \label{fig:low_pass_bode_plot}
\end{figure}


\begin{figure}[H]
    \centering
    \includegraphics[width=0.85\textwidth]{high_pass_bode_plot.png}
    \caption{High-pass filter: bode plot}
    \label{fig:high_pass_bode_plot}
\end{figure}

\subsection{Phase}

For the phase plots, use a linear vertical axis for the phase and $\log \nu$ for the horizontal axis (as in Fig.~5). Determine the breakpoint or characteristic frequency from the data plots by identifying the $-3$~dB point on the amplitude plot (also known as a Bode plot) and the \ang{45} point of the phase plot.


\begin{figure}[H]
    \centering
    \includegraphics[width=0.85\textwidth]{low_pass_phase_plot.png}
    \caption{High-pass filter: phase plot}
    \label{fig:low_pass_phase_plot}
\end{figure}


\end{document}
