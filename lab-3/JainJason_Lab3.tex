\documentclass[11pt]{article}
\usepackage[margin=1in]{geometry}
\usepackage{graphicx}
\usepackage{siunitx}
\usepackage{amsmath}
\usepackage{float}
\usepackage{caption}

\title{Lab 3: Capacitors and Time-Dependent Signals}
\author{Jason Jain \\
        Oregon State University, Department of Physics \\
        Electronics Laboratory}
\date{\today}

% Try all possible locations depending on where the build is run from
\graphicspath{{output/}{./}{./output/}{lab-3/output/}}

\begin{document}

\maketitle

\section{Time-dependent analysis of RC circuits}

We measure the time constant of the RC circuit, first calculating the theoretical value, then comparing it to the experimental value. The resistance is $R=\qty{997}{\ohm}$ and the capacitance is $C=\qty{73.5}{\nano\farad}$, giving us a theoretical time constant of $\tau = RC = \qty{73.3}{\micro\second}$. Our experimental value is found by observing the time it takes for the capacitor to reach $1/e$ of its total charge while discharging, and $1 - (1/e)$ of its total charge while charging. As seen in Figure~1, the experimental value is $\tau = \qty{62}{\micro\second}$.

The experimental and theoretical values have a percent difference of $\approx\qty{17}{\percent}$. The discrepancy is likely due to measurement uncertainties and tolerances in the component values. The experimental value is used in all further calculations in this lab. 

\begin{figure}[H]
    \centering
    \includegraphics[width=0.85\textwidth]{output/low_pass_square_1kHz.png}
    \caption{Measurement of time constant $\tau = RC$ by determining the time for the output to drop to $1/e$ of the
    maximum (in green) and to rise to $1 - (1/e)$ of the maximum (in red). Both values are equal to $\tau = \qty{62}{\micro\second}$.}
    \label{fig:low_pass_1kHz}
\end{figure}

\newpage

\section{The RC Integrator (Low-Pass Filter)}

An RC circuit is also called a low pass filter, allowing frequencies below the cutoff frequency to pass relatively unchanged. The circuit is also called an integrator because the graph of voltage out is the integral of the voltage in. The integrator filter is applied to the high frequencies and greatly reduces signal strength. 


In Figure 2, we can model the input $V_{\mathrm{in}}$ as alternating lines of $y=\pm1$, so the output integral $V_{\mathrm{out}}$ is simply $\pm\int{1\,dx}=\pm x+C$, creating the triangle wave pattern of the output voltage.


The low pass behavior is most evident for frequencies seen in Figure~2, where we can graph $V_{\mathrm{out}}$ at a frequency of $f=\qty{20}{\kilo\hertz}$ without any scaling. As we approach the \qty{200}{\kilo\hertz} frequency range in Figure 3, the output is so small that we need to scale it 10x to be noticeable at all.

\begin{figure}[H]
    \centering
    \includegraphics[width=0.85\textwidth]{output/low_pass_square_20kHz.png}
    \caption{RC integrator (low-pass filter): voltage vs time (20~kHz square input). }
    \label{fig:low_pass_20kHz}
\end{figure}

The triangle wave input of Figure 3 $V_{\mathrm{in}}$ can be modeled as alternating curves of $y=\pm mx$. The integral of the input voltage is now $\pm\int{mx\,dx}=\pm\frac{1}{2}mx^2+C$, with a concave-up parabola when the input is increasing, and a concave-down parabola when the input is decreasing. 

\begin{figure}[H]
    \centering
    \includegraphics[width=0.85\textwidth]{output/low_pass_triangle_200kHz.png}
    \caption{RC integrator (low-pass filter): voltage vs time (200~kHz triangle input). Most of the signal is filtered out due to the high input frequency, so the output voltage is scaled by a factor of 10 to make the integrator behavior more obvious. }
    \label{fig:low_pass_triangle_200kHz}
\end{figure}

\newpage

\section{The CR Differentiator (High-Pass Filter)}

We apply a \qty{510}{\hertz} square-wave signal to the CR circuit to create a high-pass filter. This circuit is called a high-pass filter because it allows frequencies above the cutoff frequency to pass relatively unchanged, while applying the differentiator behavior to the low frequencies.

\begin{figure}[H]
    \centering
    \includegraphics[width=0.85\textwidth]{output/high_pass_square_510Hz.png}
    \caption{High-pass filter: voltage vs time (510~Hz square input).}
    \label{fig:high_pass_510Hz}
\end{figure}

The waveform, seen in Figure~4, demonstrates the differentiator behavior of this circuit. The graph of voltage out is the derivative of the voltage in. We can model the input $V_{\mathrm{in}}$ as the alternating step function $y=\pm\Theta(x)$, so the output derivative is $\pm\frac{d}{dx}\Theta(x)=\pm\delta(x)$, the Dirac delta function, creating the spiking pattern of the output voltage.

\begin{figure}[H]
    \centering
    \includegraphics[width=0.85\textwidth]{output/high_pass_triangle_200Hz.png}
    \caption{CR differentiator (high-pass filter): voltage vs time (200~Hz triangle input).}
    \label{fig:high_pass_triangle_200Hz}
\end{figure}

Likewise, in Figure~5, we can model the triangle wave input $V_{\mathrm{in}}$ as alternating curves of $y=\pm mx$. The output derivative is $\pm\frac{d}{dx}mx=\pm m$, approaching a square wave pattern. As we approach lower frequencies, we lose signal strength, so the data becomes increasingly noisy. For higher frequencies in the \qty{100}{\kilo\hertz}--\qty{1}{\mega\hertz} range, the output and input would closely resemble one another, and the differentiator behavior would be absent.

\newpage

\section{Frequency response of both low-pass and high-pass filters}

We explored the frequency dependence of each filter's response to a sine wave input by varying frequency from \qty{1}{\hertz} to \qty{1}{\mega\hertz} and measuring the voltage and phase of the output $V_{\mathrm{out}}$ relative to the input $V_{\mathrm{in}}$.


The input voltage was measured peak-to-peak from the function generator at each frequency to account for changes due to the circuit and generator limitations. The experimental admittance $A_{exp}$ was found by dividing the output voltage by the input voltage at each frequency on a log scale. 

$$A_{exp}=20\log_{10} \frac{V_{out}}{V_{in}}$$ 

The theoretical admittance is found with the known formulas 
for high and low pass filters.

\begin{align*}
    A_{high}(f)&=20\log_{10} \frac{2\pi\,f / \omega_c}{\sqrt{1+2\pi\,f / \omega_c}} \\
    A_{low}(f)&=20\log_{10} \frac{1}{\sqrt{1+2\pi\,f / \omega_c}}
\end{align*}

For each filter we plotted admittance vs frequency together with the theoretical curves, using the time constant of $\tau=\qty{62}{\micro\s}$ to find the theoretical characteristic \textit{angular} frequency of $\omega_c=1 / \tau=\qty{16}{\kilo\hertz}$.

\begin{figure}[H]
    \centering
    \includegraphics[width=0.85\textwidth]{output/low_pass_bode_plot.png}
    \caption{Low-pass filter: Bode plot. The blue vertical line marks the experimental cutoff $f_{c,\mathrm{exp}}$.}
    \label{fig:low_pass_bode_plot}
\end{figure}

\begin{figure}[H]
    \centering
    \includegraphics[width=0.85\textwidth]{output/high_pass_bode_plot.png}
    \caption{High-pass filter: Bode plot. The blue vertical line marks the experimental cutoff $f_{c,\mathrm{exp}}$.}
    \label{fig:high_pass_bode_plot}
\end{figure}


Phase difference $\phi(\nu)$ between output and input was measured using the cursor function on the oscilloscope to measure the time difference $\delta t$ between the peaks of each waveforms. A phase lag (output peak after the input peak) corresponds to a negative phase. The experimental and theoretical phases are calculated with the formula 

$$\phi_{\mathrm{exp}}=2\pi t/T=2\pi tf$$
$$\phi_{\mathrm{theoretical}}=\pm\arctan(2\pi f/\omega_c)$$

\begin{figure}[H]
    \centering
    \includegraphics[width=0.85\textwidth]{output/low_pass_phase_plot.png}
    \caption{Low-pass filter: phase plot. The blue vertical line marks $f_{c,\mathrm{exp}}$.}
    \label{fig:low_pass_phase_plot}
\end{figure}


The experimental characteristic frequency $f_{c,\mathrm{exp}}$ is obtained by linear extrapolation in $\log f$ from the data point closest to the \qty{-3}{\decibel} point on the Bode plot or the \ang{45} point on the phase plot; the blue vertical line on each figure marks this extrapolated value. Slight discrepancies between $f_{c,\mathrm{exp}}$ and the theoretical $f_c$ are expected due to the sensitive nature of the log scale; despite that, the measured phase and admittance data fit the theoretical curves well.



\end{document}