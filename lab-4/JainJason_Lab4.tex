\documentclass[11pt]{article}
\usepackage[margin=0.75in]{geometry}
\usepackage{graphicx}
\usepackage{siunitx}
\usepackage{amsmath}
\usepackage{float}
\usepackage{caption}

\title{Lab 3: Inductors and Time-Dependent Signals}
\author{Jason Jain \\
        Oregon State University, Department of Physics \\
        Electronics Laboratory}
\date{\today}

\graphicspath{{output/}}

\begin{document}

\maketitle

\section{Time dependent analysis of $RL$ Circuits}

The resistance is $\qty{997}{\ohm}$ and the inductance is $\qty{21.4}{\milli\henry}$

% \ref{fig:low_pass_1kHz}
\begin{figure}[H]
    \centering
    \includegraphics[width=0.85\textwidth]{low_pass_square_1kHz.png}
    \caption{Measurement of time constact $\tau = RC$ by determining the time for the output to drop to $1/e$ of the
    maximum (in green)and to rise to $1 - (1/e)$ of the maximum (in red). Both values are equal to $\tau = \qty{62}{\micro\second}$.}
    \label{fig:low_pass_1kHz}
\end{figure}

\newpage

\section{The RC Integrator (Low-Pass Filter)}

- Use the circuit of Part 1 and apply a 20 kHz square-wave signal.

- Explain how the observed waveform is consistent with the concept of an RC circuit behaving as an integrator.

- Over what frequency range does the circuit behave as an integrator, that is, is capable of producing a triangle-wave output from a square-wave input? Explain why this circuit is also known as a low-pass filter.


\newpage

\section{The CR Differentiator (High-Pass Filter)}

- Explain how the observed waveform is consistent with the concept of an CR circuit behaving as an differentiator.


- Vary the frequency of the square wave input from 1 Hz to 1 MHz and describe the behavior of this circuit. Does it ever appear to behave as an integrator or a differentiator? Explain why this circuit is also known as a high-pass filter.



\newpage

\section{Frequency response of both low-pass and high-pass filters}



\end{document}
