\documentclass[11pt]{article}
\usepackage[margin=1in]{geometry}
\usepackage{graphicx}
\usepackage{siunitx}
\usepackage{amsmath}
\usepackage{float}
\usepackage{caption}
\usepackage{subcaption}


\title{Lab 1: Exploring Relationship between Current, Voltage, and Resistance}
\author{Jason Jain \\ 
        Oregon State University, Department of Physics \\
        Electrics Laboratory}
\date{\today} % Change to specific date if needed

\begin{document}

\maketitle

\section{Current v.s. Voltage Characteristics of Circuit Elements}

\subsection{I-V Curve for a Resistor}


\begin{figure}[H]
    \centering
    \includegraphics[width=0.7\textwidth]{output/resistor_current_vs_voltage.png}
    \caption{Best fit line illustrates Ohm's Law for a resistor.}
    \label{fig:resistor_iv}
\end{figure}


Here we see experimental data values for current vs voltage for a resistor, with a best fit line demonstrating Ohm's Law.

% Put any extra analysis here to explain Fig. 1
% Example: The linear relationship between current and voltage demonstrates Ohm's law...

\pagebreak

\subsection{I-V Curves for Diodes}


\begin{figure}[H]
    \centering
    \includegraphics[width=0.7\textwidth]{output/led_current_vs_voltage.png}
    \caption{Experimental values for LED and diode current/voltage closely follow a logarithmic relationship.}
    \label{fig:led_iv}
\end{figure}


Here we see experimental data values for current vs voltage for a diode, with a logarithmic best fit demonstrating the relationship between voltage and current. Normally, the current would be on the y-axis and voltage on the x-axis, but this makes the data hard to see at low voltages, so I flipped the axes. I plotted the diode and both the LEDs on the same graph, to observe how they all obey the same functional dependence between voltage $V$, and current $I$:

$$V(I)=V_{T}\,\ln\left( \frac{I}{I_{0}}+1 \right)$$


% Put any extra analysis here to explain Figs. 2 and 3
% Example: The forward bias behavior shows...

\pagebreak

\subsection{Output Voltage of Voltage Dividers}



\begin{figure}[H]
    \centering
    \includegraphics[width=0.7\textwidth]{output/voltage_divider.png}
    \caption{The experimental values closely match the theoretical prediction for a voltage divider circuit.}
    \label{fig:voltage_divider}
\end{figure}


We previously derived the voltage $V_L$ for a voltage divider as a function of load resistance $R_L$. In Figure 3, we see the experimental data is very close to the theoretical prediction. We previously derived the equation for $V_L$ as:

$$
V_{L}=\frac{R_LR_2 V_S}{(R_2 R_L + R_1 R_L + R_1 R_2)}
$$



% Put analysis, equations, etc here, comparing with the problem set results is essential 
% but no need to reproduce the derivation
% Example: The voltage divider equation predicts:
% \begin{equation}
%     V_{out} = V_{in} \frac{R_2}{R_1 + R_2}
% \end{equation}

\newpage

\section{Part II: Calculating total resistance of a complex circuit}
% (e.g. Open Circuit Voltage, and Short Circuit Current Analysis)

Here we see a circuit diagram where calculating the total resistance may not be straightforward. We measure the open circuit voltage and short circuit current to calculate the total resistance of the circuit.

\begin{figure}[H]
    \centering
    \includegraphics[width=0.7\textwidth]{output/circuit.png}
    \caption{Computer-generated circuit diagram.}
\end{figure}


% Put analysis, equations, etc here
% Example: The open circuit voltage was measured as V_OC = ...
% The short circuit current was measured as I_SC = ...

% No graphs are needed, but be sure to record all values and answer ALL questions from the report.

\textbf{Open Circuit Voltage:} $V_{OC} = \qty{0.661}{\volt} $ 

\textbf{Short Circuit Current:} $I_{SC} = \qty{10.23}{\milli\ampere} $

\textbf{Discussion:} Please note that this is the closest we could get to the short circuit current with our equipment, because the ammeter has some minimum resistance. The total resistance is then,

$$R_T=\frac{0.01023}{0.661-0.580}=\qty{0.126}{\ohm}$$


\begin{figure}[H]
    \centering
    \includegraphics[width=0.7\textwidth]{output/total_resistance.png}
    \caption{Graphical depiction of open circuit voltage and short circuit current. The total resistance is the absolute value of the slope.}
    \label{fig:circuit}
\end{figure}



\end{document}